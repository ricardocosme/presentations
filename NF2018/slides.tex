%!TEX program = xelatex

% Name           : hsrm-beamer-demo.sty
% Author         : Benjamin Weiss (benjamin.weiss@kreatiefton.de)
% Version        : 0.4
% Created on     : 05.05.2013
% Last Edited on : 24.03.2014
% Copyright      : Copyright (c) 2013-2014 by Benjamin Weiss. All rights reserved.
% License        : This file may be distributed and/or modified under the
%                  GNU Public License.
% Description    : HSRM beamer theme demonstration. Also includes a short 
%                  Tutorial regarding the beamer class.

\documentclass[compress]{beamer}
%--------------------------------------------------------------------------
% Common packages
%--------------------------------------------------------------------------
\usepackage[portuguese]{babel}
\usepackage{pgfplots}
\usepackage{array}
\usepackage{graphicx}
\usepackage{multicol}
% Erweiterte Tabellenfunktionen
\usepackage{tabularx,ragged2e}
\usepackage{booktabs}
% Listingserweiterung
\usepackage{pgf, pgffor}
\usepackage{listings}
\usepackage{dirtree}
\usepackage{lstlinebgrd}

\usepackage{hyperref}
\usepackage{fontspec}
\usepackage{siunitx}
\sisetup{math-micro=\text{µ},text-micro=µ}

\makeatletter
%%%%%%%%%%%%%%%%%%%%%%%%%%%%%%%%%%%%%%%%%%%%%%%%%%%%%%%%%%%%%%%%%%%%%%%%%%%%%%
%
% \btIfInRange{number}{range list}{TRUE}{FALSE}
%
% Test in int number <number> is element of a (comma separated) list of ranges
% (such as: {1,3-5,7,10-12,14}) and processes <TRUE> or <FALSE> respectively

\newcount\bt@rangea
\newcount\bt@rangeb

\newcommand\btIfInRange[2]{%
    \global\let\bt@inrange\@secondoftwo%
    \edef\bt@rangelist{#2}%
    \foreach \range in \bt@rangelist {%
        \afterassignment\bt@getrangeb%
        \bt@rangea=0\range\relax%
        \pgfmathtruncatemacro\result{ ( #1 >= \bt@rangea) && (#1 <= \bt@rangeb) }%
        \ifnum\result=1\relax%
            \breakforeach%
            \global\let\bt@inrange\@firstoftwo%
        \fi%
    }%
    \bt@inrange%
}
\newcommand\bt@getrangeb{%
    \@ifnextchar\relax%
        {\bt@rangeb=\bt@rangea}%
        {\@getrangeb}%
}
\def\@getrangeb-#1\relax{%
    \ifx\relax#1\relax%
        \bt@rangeb=100000%   \maxdimen is too large for pgfmath
    \else%
        \bt@rangeb=#1\relax%
    \fi%
}

%%%%%%%%%%%%%%%%%%%%%%%%%%%%%%%%%%%%%%%%%%%%%%%%%%%%%%%%%%%%%%%%%%%%%%%%%%%%%%
%
% \btLstHL<overlay spec>{range list}
%
% TODO BUG: \btLstHL commands can not yet be accumulated if more than one overlay spec match.
% 
\newcommand<>{\btLstHL}[1]{%
  \only#2{\btIfInRange{\value{lstnumber}}{#1}{\color{yellow!20}\def\lst@linebgrdcmd{\color@block}}{\def\lst@linebgrdcmd####1####2####3{}}}%
}%
\makeatother

\lstdefinestyle{emph}{
  % keywordstyle=\color{orange}\bfseries
  linebackgroundcolor=\color{orange!15}
}

\lstset{
literate={~} {$\sim$}{1}, % set tilde as a literal (no process)
escapeinside=@@,
language=c++,
% basicstyle=\small,
basicstyle=\normalsize\sffamily,
keywordstyle=\color{hsrmSec1Dark}\bfseries,
stringstyle=\color{red},
commentstyle=\color{gray},
morecomment=[l][\color{magenta}]{\#},
numbers=left,
numberstyle=\tiny\ttfamily,
stepnumber=1,
showspaces=false,
showstringspaces=false,
showtabs=false,
breaklines=true,
% frame=tb,
framerule=0.5pt,
tabsize=2,
framexleftmargin=0.5em,
framexrightmargin=0.5em,
xleftmargin=0.5em,
xrightmargin=0.5em,
}

%--------------------------------------------------------------------------
% Load theme
%--------------------------------------------------------------------------
\usetheme{hsrm}

% \usepackage{dtklogos} % must be loaded after theme
\usepackage{tikz}
\usetikzlibrary{mindmap,backgrounds}

%--------------------------------------------------------------------------
% General presentation settings
%--------------------------------------------------------------------------
\title{Coruja}
\subtitle{Uma alternativa ao padrão observador em C++11}
\date{\today}
\author{Ricardo Cosme}
\institute{Instituto Tecgraf de Desenvolvimento de \\ Software Técnico-Científico da PUC-Rio\\{\Medium Tecgraf/PUC-Rio}}

%--------------------------------------------------------------------------
% Notes settings
%--------------------------------------------------------------------------
\setbeameroption{show notes}

\begin{document}
%--------------------------------------------------------------------------
% Titlepage
%--------------------------------------------------------------------------

\maketitle

%\begin{frame}[plain]
%	\titlepage
%\end{frame}

\begin{frame}{Contexto}
	\begin{itemize}
		\item Há 2 anos trabalhando com um software desktop para geofísicos da Petrobrás
      \begin{itemize}
      \item Visualização 2D e 3D, CUDA, análise de dados e projeções.
        \begin{itemize}
          \item Múltiplas visões reativas de um dado
        \end{itemize}
      \item C++11 - Programação genérica(GP) e funcional(FP).
      \end{itemize}
	\end{itemize}
\end{frame}

%--------------------------------------------------------------------------
% Table of contents
%--------------------------------------------------------------------------
% \section*{Seções}
% \begin{frame}{Seções}
% 	% hideallsubsections ist empfehlenswert für längere Präsentationen
% 	\tableofcontents[hideallsubsections]
% \end{frame}

%--------------------------------------------------------------------------
% Content
%--------------------------------------------------------------------------
\section{Padrão observador}

\begin{frame}{Problema}
A alteração do estado saliente de um objeto deve {\Medium imediatamente} atualizar zero ou mais visões do mesmo. 
\end{frame}

\begin{frame}{Requisitos}
  A solução deve considerar:
	\begin{itemize}
		\item<1->Baixo acoplamento entre as partes.
		\item<2->Número dinâmico de visões.
		\item<3->Contexto \textit{singlethread}. (Widgets e OpenGL)
		\item<4->\textit{push model}: atualizações imediatas via \textit{callback}
	\end{itemize}
\end{frame}

\begin{frame}[containsverbatim]{Padrão observador}
\begingroup
\setbeamercolor{block title}{bg=hsrmSec2Dark}
\setbeamercolor{block body}{bg=hsrmSec2}
\begin{block}{Objetivo}
	Definir uma dependência one-to-many entre objetos de forma que quando o estado de um objeto é alterado, todas as suas dependências são notificadas e atualizadas automaticamente. \footnote{Design Patterns: Elements Of Reusable Object-Oriented Software (1994, Addison Wesley)}
\end{block}
\endgroup
\end{frame}

\subsection{Modelo}
\begin{frame}[fragile,t]{Observer}
\begin{lstlisting}[
    gobble=0,
    linebackgroundcolor={%
      \btLstHL<1>{0}%
      \btLstHL<2>{1}%
      \btLstHL<3>{2}%
      \btLstHL<4->{3}%
    }]
struct observer {
  virtual ~observer() = default;
  virtual void update(Subject&);
};
\end{lstlisting}
\only<2>{
Pouco flexível.
\begin{itemize}
\item Preciso somente de uma função.
\end{itemize}
}
\only<3>{
Polimorfismo através de herança.
\begin{itemize}
  \item{Restringe demasiadamente o design de quem usa.}
\end{itemize}
}
\only<4->{
\begin{enumerate}
  \item<4->{O que mudou?}
  \item<5->{Uma única função para todas as reações (complexidade).}
\end{enumerate}
}
\end{frame}

\begin{frame}[fragile,t]{Subject (Observable)}
\begin{lstlisting}[
    gobble=0,
    linebackgroundcolor={%
      \btLstHL<1>{0}%
      \btLstHL<2>{2}%
      \btLstHL<3>{3,4}%
      \btLstHL<4-5>{5}%
    }]
struct subject {
  virtual ~subject() = default;
  virtual void attach(observer&);
  virtual void detach(observer&);
  virtual void notify();

  list<observer*> observers;
};
\end{lstlisting}
\only<2>{
Polimorfismo através de herança.
\begin{itemize}
  \item{Restringe demasiadamente o design de quem usa.}
\end{itemize}
}
\only<3>{Demasiadamente operacional (baixo nível).}
\only<4-5>{
\begin{enumerate}
\item<4->{Notificação global.}
\begin{itemize}
  \item<4->{O que mudou?}
\end{itemize}
\item<5->{Notificação manual.}
\end{enumerate}
}
\end{frame}

\subsection{Signals\&Slots}
\begin{frame}[fragile,t]{Signals\&Slots}
\lstset{numbers=none}
\begin{lstlisting}
//Signal (Observable)
signal<void(string)> name_sig;
\end{lstlisting}

\only<2->{
\begingroup
\setbeamercolor{block title}{bg=hsrmSec2Dark}
\setbeamercolor{block body}{bg=hsrmSec2}
\begin{block}{Soluciona}
\begin{itemize}
\item<2->{Composição ao invés de herança}
\item<3->{Específico para um atributo: \texttt{name}}
\end{itemize}
\end{block}
\endgroup
}
\end{frame}

\begin{frame}[fragile,t]{Signals\&Slots}
\lstset{numbers=none}
\begin{lstlisting}[
    gobble=0,
    linebackgroundcolor={%
      \btLstHL<1>{2-3}%
      \btLstHL<2>{2-3}%
      \btLstHL<3>{5}%
      \btLstHL<4>{7}%
    }]
//Slot (Observer) attach() 
auto conn = name_sig.connect
  ([](string name){do_something(name);});

name_sig("abc"); //notify() 

conn.disconnect(); //detach()
\end{lstlisting}

\begingroup
\setbeamercolor{block title}{bg=hsrmSec2Dark}
\setbeamercolor{block body}{bg=hsrmSec2}
\begin{block}{Soluciona}
\begin{itemize}
\item<1->{\emph{Observer} é simplesmente um function object.}
\item<2->{Função só se preocupa com uma coisa: \texttt{name}.}
\end{itemize}
\end{block}
\endgroup
\end{frame}

\begin{frame}[fragile,t]{Signals\&Slots}
\begin{alertblock}{É suficiente?}
  \begin{itemize}
    \item<1-> Inversão de Controle(IoC) (callbacks)
    \item<2-> Notificação manual
    \item<3-> Código boilerplate
    \item<4-> Dependência da ordem de registro de slots
    \item<5-> Complexidade em compor observáveis (sinais)
  \end{itemize}
\end{alertblock}
\end{frame}

\section{Objetos observáveis}

\subsection{Signals&Slots}
\begin{frame}[fragile,t]{Problema \#1}
\lstset{basicstyle=\footnotesize\sffamily}
\begin{lstlisting}
struct person_t {
  string first_name, surname;
};
\end{lstlisting}
\only<2>{Problema \#1: Reagir a alteração de first\_name ou surname}
\end{frame}

\subsection{Signals&Slots}
\begin{frame}[fragile,t]{Problema \#1 - Solução Ad Hoc}
\lstset{basicstyle=\footnotesize\sffamily}
\begin{lstlisting}[
    gobble=0,
    linebackgroundcolor={%
      \btLstHL<1>{2,8,12}%
    }]
struct person_t {
  //setter
  void first_name(string v) {
      _first_name = move(v);
      _change_first_name(_first_name);
  }

  //getter
  const string& first_name() const noexcept
  { return _first_name; }

  //connect
  template<typename F>
  void change_first_name(F&& f)
  { _change_first_name.connect(forward<F>(f)); }
private:
  signal<void(const string&)> _change_first_name;
  string _first_name;
};
\end{lstlisting}
\end{frame}

\begin{frame}[fragile,t]{Problema \#1 - Solução Ad Hoc}
\lstset{basicstyle=\tiny\sffamily}
\begin{lstlisting}
struct person_t {
  void first_name(string v) {
      _first_name = move(v);
      _change_first_name(_first_name);
  }

  const string& first_name() const noexcept
  { return _first_name; }
  
  template<typename F>
  void change_first_name(F&& f)
  { _change_first_name.connect(forward<F>(f)); }

  void surname(string v) {
      _surname = move(v);
      _change_surname(_surname);
  }

  const string& surname() const noexcept
  { return _surname; }
  
  template<typename F>
  void change_surname(F&& f)
  { _change_surname.connect(forward<F>(f)); }
private:
  signal<void(const string&)> _change_first_name, _change_surname;
  string _first_name, _surname;
};
\end{lstlisting}
\end{frame}

\begin{frame}[fragile,t]{Problema \#1 - Solução Coruja}
\lstset{basicstyle=\footnotesize\sffamily}
\begin{lstlisting}[
    gobble=0,
    linebackgroundcolor={%
      \btLstHL<1->{2}%
    }]
struct person_t {
  coruja::object<string> first_name, surname;
};
\end{lstlisting}
\only<2>{
\begin{table}[]
  \footnotesize
  \begin{tabularx}{\linewidth}{l>{\raggedright}X}
    \toprule
    \textbf{Ad hoc}			& \textbf{Coruja}   \tabularnewline 
    \midrule
    \texttt{surname(v);}		& \texttt{surname = v;}  \tabularnewline
    \texttt{auto v = surname();}	& \texttt{auto v = surname.observed();}  \tabularnewline
    \texttt{change\_surname.connect(f);}& \texttt{surname.after\_change(f);}  \tabularnewline
    \bottomrule
  \end{tabularx}
  \label{tab:options}
\end{table}
}
\end{frame}

\begin{frame}[fragile,t]{Problema \#1 - Solução Coruja}
\lstset{basicstyle=\footnotesize\sffamily}
\begingroup
\setbeamercolor{block title}{bg=hsrmSec2Dark}
\setbeamercolor{block body}{bg=hsrmSec2}
\begin{block}{Observable}
Notifica imediatamente observadores interessados em uma ação específica de alteração de estado de um objeto de tipo T.
\end{block}
\endgroup
\lstset{
basicstyle=\footnotesize\sffamily,
numbers=none
}
\begin{table}[]
  \footnotesize
  \begin{tabularx}{\linewidth}{l>{\raggedright}X}
    \toprule
    \textbf{Requisitos}   \tabularnewline 
    \midrule
    \begin{lstlisting}^^J
      Observable::observed\_t^^J
    \end{lstlisting}
    \tabularnewline
    \tabularnewline
    \begin{lstlisting}^^J
      observed\_t observed() const noexcept^^J
    \end{lstlisting}
    \tabularnewline
    \tabularnewline
    \begin{lstlisting}^^J
      //There has to be at least one observable action^^J
      //FunctionObject must be CopyConstructible^^J
      template<typename FunctionObject>^^J
      connection action(FunctionObject)^^J
    \end{lstlisting} 
    \tabularnewline
    \tabularnewline
    DefaultConstructible
    \tabularnewline
    \bottomrule
  \end{tabularx}
  \label{tab:options}
\end{table}
\end{frame}

\begin{frame}[fragile,t]{Problema \#1 - Solução Coruja}
\lstset{basicstyle=\footnotesize\sffamily}
\begingroup
\setbeamercolor{block title}{bg=hsrmSec2Dark}
\setbeamercolor{block body}{bg=hsrmSec2}
\begin{block}{ObservableObject}
Refina o conceito \emph{Observable} com uma ação observável geral \texttt{after\_change}
\end{block}
\endgroup
\lstset{
basicstyle=\footnotesize\sffamily,
numbers=none
}
\begin{table}[]
  \footnotesize
  \begin{tabularx}{\linewidth}{l>{\raggedright}X}
    \toprule
    \textbf{Requisitos}   \tabularnewline 
    \midrule
    \begin{lstlisting}^^J
      //FunctionObject should be void(const Observable::observed_t\&)^^J
      template<typename FunctionObject>^^J
      after_change_connection_t after_change(FunctionObject)^^J
    \end{lstlisting}
    \tabularnewline
    \tabularnewline
    \begin{lstlisting}^^J
      Observable::after_change_connection_t^^J
    \end{lstlisting}
    \tabularnewline
    \bottomrule
  \end{tabularx}
  \label{tab:options}
\end{table}
\end{frame}

\begin{frame}[fragile,t]{Problema \#1 - Solução Coruja}
\lstset{basicstyle=\tiny\sffamily}
\begin{lstlisting}[
    %% emph={array,connection_t,conns,thickness,marker,print_thickness,print_new_thickness},
    %% emphstyle={\color{hsrmSec3CompDark}},
    gobble=0,
    linebackgroundcolor={%
      \btLstHL<1>{1-2}%
      \btLstHL<2>{6,25}%
      \btLstHL<3>{8,21,24}%
      \btLstHL<4>{11,13}%
      \btLstHL<5>{15,16,18,19}%
    }]
template<class T, class Derived_ = void, template <typename> class Signal = signal>  
class object /* Models ObservableObject */ {
    using Derived = typename std::conditional<
        std::is_same<Derived_, void>::value, object, Derived_>::type;

    using after_change_t=Signal<void(Derived&)>;    
public:
    using observed_t = T;
    using after_change_connection_t = typename after_change_t::connection_t;
    
    object() = default;
    
    explicit object(observed_t observed) : _observed(move(observed)) {}
    
    object(object&&)
      noexcept(is_nothrow_move_constructible<observed_t>::value) {/*...*/}
    
    object& operator=(object&&)
      noexcept(is_nothrow_move_assignable<observed_t>::value) {/*...*/}
    
    const observed_t& observed() const noexcept { return _observed; }
    /* code */
private:
    observed_t _observed;    
    after_change_t _after_change;
};        
\end{lstlisting}
\end{frame}

\begin{frame}[fragile,t]{Problema \#1 - Solução Coruja}
\lstset{basicstyle=\tiny\sffamily}
\begin{lstlisting}[
    %% emph={array,connection_t,conns,thickness,marker,print_thickness,print_new_thickness},
    %% emphstyle={\color{hsrmSec3CompDark}},
    gobble=0,
    linebackgroundcolor={%
      \btLstHL<1>{7-10}%
      \btLstHL<2>{12-16}%
    }]
template<class T, class Derived_ = void, template <typename> class Signal = signal>  
class object /* Models ObservableObject */ {
public:

    /* code */

    template<typename F> //void(Derived&)
    enable_if_is_invocable_t<after_change_connection_t, F, Derived&>
    after_change(F&& f)
    { return _after_change.connect(forward<F>(f)); }
    
    template<typename F> //void(const observed_t&)
    enable_if_is_invocable_t<after_change_connection_t, F, const observed_t&>
    after_change(F&& f)
    { return _after_change.connect
            (detail::lift_to_observable{forward<F>(f)}); }
};        
\end{lstlisting}
\end{frame}

\begin{frame}[fragile,t]{Problema \#1 - Solução Coruja}
\lstset{basicstyle=\scriptsize\sffamily}
\begin{lstlisting}[
    %% emph={array,connection_t,conns,thickness,marker,print_thickness,print_new_thickness},
    %% emphstyle={\color{hsrmSec3CompDark}},
    gobble=0,
    linebackgroundcolor={%
      \btLstHL<1->{3-6}%
    }]
template<typename F>    
struct lift_to_observable_impl {
    template<typename... ObservableObjects>
    auto operator()(ObservableObjects&&... objects)
    CORUJA_DECLTYPE_AUTO_RETURN
    ( f(objects.observed()...) )

    F f;
};
\end{lstlisting}
\begin{onlyenv}<2->
\begin{lstlisting}[numbers=none]
//ObservableObject<T1> vs ObservableObject<T2>
//T vs ObservableObject<T>
//ObservableObject<T> vs T    
operator{==,!=,<,>,<=,>=}
\end{lstlisting}
\footnotesize
\begin{itemize}
\item O sinal é desconsiderado (sujeito a alterações)
\end{itemize}
\end{onlyenv}
\end{frame}

\begin{frame}[fragile,t]{Problema \#1 - Solução Ad Hoc}
\lstset{basicstyle=\footnotesize\sffamily}
\begin{lstlisting}[
    emph={array,connection_t,conns,thickness,marker,print_thickness,print_new_thickness},
    % emphstyle={\color{hsrmRedDark}},
    emphstyle={\color{hsrmSec3CompDark}},
    gobble=0,
    linebackgroundcolor={%
      \btLstHL<1>{1-2}%
      \btLstHL<2>{5-6}%
      \btLstHL<3>{4,11}%
    }]
auto print_fullname = [&p](string)
  { cout << p.first_name() + p.surname(); };

array<any_connection, 2> conns {
   p.change_first_name(print_fullname),
   p.change_surname(print_fullname),
};

p.first_name("jimmy");

for(auto& c : conns) c.disconnect();
\end{lstlisting}
\end{frame}

%% \begin{frame}[fragile,t]{Problema \#1 - Solução Ad Hoc}
%% \lstset{basicstyle=\footnotesize\sffamily}
%% \begin{lstlisting}[
%%     emph={array,connection_t,conns,thickness,marker,print_thickness,print_new_thickness},
%%     % emphstyle={\color{hsrmRedDark}},
%%     emphstyle={\color{hsrmSec3CompDark}},
%%     gobble=0,
%%     linebackgroundcolor={%
%%       \btLstHL<1>{3,4}%
%%       \btLstHL<2>{10-12}%
%%       \btLstHL<3>{14,15}%
%%     }]
%% template<typename F>
%% struct fullname_impl {
%%     void operator()(string s)
%%     { f(p.first_name() + p.surname()); }
    
%%     person_t& p;
%%     F f;
%% };

%% template<typename F>
%% inline fullname_impl<F> fullname(person_t& p, F&& f)
%% { return fullname_impl<F>{p, forward<F>(f)}; }

%% auto print_fullname =
%%   fullname(p, [](string s){ cout << s; });
%% \end{lstlisting}
%% \end{frame}

\begin{frame}[fragile,t]{Problema \#1 - Solução Coruja}
\lstset{basicstyle=\footnotesize\sffamily}
\begin{lstlisting}[
    emph={array,connection_t,conns,thickness,marker,print_thickness,print_new_thickness},
    % emphstyle={\color{hsrmRedDark}},
    emphstyle={\color{hsrmSec3CompDark}},
    gobble=0,
    linebackgroundcolor={%
      \btLstHL<1>{1}%
      \btLstHL<2>{3}%
      \btLstHL<3>{5,9}%
  }]
auto print = [](string s){cout << s;};  

auto fullname = p.first_name + p.surname;

auto c = fullname.after_change(print);

p.first_name = "jimmy";

c.disconnect();
\end{lstlisting}
\end{frame}

\begin{frame}[fragile,t]{Problema \#1 - Widget::enable}
\lstset{basicstyle=\scriptsize\sffamily}
\begin{lstlisting}[
    emph={array,connection_t,conns,thickness,marker,print_thickness,print_new_thickness},
    % emphstyle={\color{hsrmRedDark}},
    emphstyle={\color{hsrmSec3CompDark}},
    gobble=0,
    linebackgroundcolor={%
  }]
//AdHoc
void Widget::enable(bool)

auto conn = name.after_change
  ([&w](string s){ w.enable(!s.empty()); });

conn.disconnect();
\end{lstlisting}
\begin{onlyenv}<2->
  \begin{itemize}
  \item<2-> Inversão de controle (IoC).
    \begin{itemize}
    \item<2->{Bug se \texttt{\&w} mudar.}
    \end{itemize}
  \item<3-> Tenho que gerenciar a conexão.
  \end{itemize}
\end{onlyenv}
\end{frame}

\begin{frame}[fragile,t]{Problema \#1 - Solução Coruja}
\begingroup
\setbeamercolor{block title}{bg=hsrmSec2Dark}
\setbeamercolor{block body}{bg=hsrmSec2}
\begin{block}{ObservableView}
  Refina o conceito \emph{Observable}, é \emph{Semiregular} e operações de cópia, move e atribuição são O(1).
\end{block}
\endgroup
\lstset{basicstyle=\scriptsize\sffamily}
\begin{lstlisting}[
    emph={array,connection_t,conns,thickness,marker,print_thickness,print_new_thickness},
    % emphstyle={\color{hsrmRedDark}},
    emphstyle={\color{hsrmSec3CompDark}},
    gobble=0,
    linebackgroundcolor={%
      \btLstHL<1>{1,2}%
      \btLstHL<2>{8}%
      \btLstHL<3>{7}%
  }]
//T2 F(T1)
//O<T2> transform(O<T1>, F)

template<typename ObservableObject, typename F>
inline enable_if_t<
    is_observable_object<ObservableObject>::value,
    detail::transform_object_t<ObservableObject, F>> //O<T2>
transform(ObservableObject&& o, F&& f)
\end{lstlisting}
\end{frame}

\begin{frame}[fragile,t]{Problema \#1 - Widget::enable}
\lstset{basicstyle=\scriptsize\sffamily}
\begin{lstlisting}[
    emph={array,connection_t,conns,thickness,marker,print_thickness,print_new_thickness},
    % emphstyle={\color{hsrmRedDark}},
    emphstyle={\color{hsrmSec3CompDark}},
    gobble=0,
    linebackgroundcolor={%
      \btLstHL<1>{2,3}%
      \btLstHL<2>{5}%
      \btLstHL<3>{5}%
      \btLstHL<4>{5}%
  }]
//ObservableObject::observed_t is bool
template<typename ObservableObject>  
void Widget::enable(ObservableObject&&)

w.enable(transform(name, [](string s){return !s.empty();}));
\end{lstlisting}
\begin{onlyenv}<3->
  \begin{itemize}
  \item<3-> Não há inversão de controle (IoC)
  \item<4-> Burocracias gerenciadas pelo widget
    \begin{itemize}
    \item<4-> Conexão
    \item<4-> Ponteiro para o widget
    \end{itemize}
  \end{itemize}
\end{onlyenv}
\end{frame}

\begin{frame}[fragile,t]{Problema \#1 - Solução Coruja}
\lstset{basicstyle=\tiny\sffamily}
\begin{lstlisting}[
    emph={array,connection_t,conns,thickness,marker,print_thickness,print_new_thickness},
    % emphstyle={\color{hsrmRedDark}},
    emphstyle={\color{hsrmSec3CompDark}},
    gobble=0,
    linebackgroundcolor={%
      \btLstHL<1>{1,2}%
      \btLstHL<2>{22,23}%
      \btLstHL<3>{19-20}%
      \btLstHL<4>{12-17}%
  }]
template<typename From, typename Transform>
struct transform_object /* Models ObservableView */ {
    using observed_t = result_of_t<Transform(detail::observed_t<From>)>;;
    using after_change_connection_t = typename From::after_change_connection_t;

    transform_object() = default;
    transform_object(From from, Transform transform)
        : _transform(move(transform))
        , _from(move(from))
    {}
    
    template<typename F>
    after_change_connection_t after_change(F&& f) {
        return _from.after_change
            (detail::after_change_cbk<From, Transform, F>
             {_transform, forward<F>(f)});
    }
            
    observed_t observed() const noexcept
    { return _transform(_from.observed()); }
private:    
    ranges::semiregular_t<Transform> _transform;
    From _from;
};
\end{lstlisting}
\end{frame}

\begin{frame}[fragile,t]{Problema \#1 - Solução Coruja}
\lstset{basicstyle=\scriptsize\sffamily}
\begin{lstlisting}[
    emph={array,connection_t,conns,thickness,marker,print_thickness,print_new_thickness},
    % emphstyle={\color{hsrmRedDark}},
    emphstyle={\color{hsrmSec3CompDark}},
    gobble=0,
    linebackgroundcolor={%
      \btLstHL<1>{10,11}%
  }]
namespace coruja { namespace detail {

template<typename From, typename Transform, typename F>    
struct after_change_cbk : protected Transform {
    after_change_cbk(Transform t, F f)
        : Transform(move(t))
        , _f(move(f))
    {}
    
    void operator()(const typename From::observed_t& from)
    { _f(Transform::operator()(from)); }

    F _f;
};
        
}}
\end{lstlisting}
\end{frame}

\begin{frame}[fragile,t]{Problema \#1 - Solução Coruja}
\lstset{basicstyle=\scriptsize\sffamily}
\begin{lstlisting}[
    emph={array,connection_t,conns,thickness,marker,print_thickness,print_new_thickness},
    % emphstyle={\color{hsrmRedDark}},
    emphstyle={\color{hsrmSec3CompDark}},
    gobble=0,
    linebackgroundcolor={%
      \btLstHL<1>{1-2}%
      \btLstHL<2>{4-6}%
  }]
template<typename Transform, typename... Objects>
auto lift(Transform&&, Objects&...)
    
template<typename O1, typename O2>
auto operator+(O1& o1, O2& o2)
{ return lift(Plus{}, o1, o2); }
\end{lstlisting}
\end{frame}

\begin{frame}[fragile,t]{Problema \#1 - Solução Coruja}
\lstset{basicstyle=\tiny\sffamily}
\begin{lstlisting}[
    linebackgroundcolor={%
      \btLstHL<1>{3}%
      \btLstHL<2>{6-7}%
      \btLstHL<3>{13}%
      \btLstHL<4>{15-16}%
      \btLstHL<5>{18}%
  }]
template<typename T, typename Transform, typename... Objects>
class lift_object : view_base {
  using From = boost::fusion::vector<Objects...>;
public:
  using observed_t = T;
  using after_change_connection_t = connections<
    typename remove_reference_t<Objects>::after_change_connection_t...>;
  
  template<typename F>
  after_change_connection_t after_change(F&& f) {
    using namespace boost::fusion;
    using conns_t = typename after_change_connection_t::type;
    using Obj2Conn = vector<From&, conns_t&>;

    conns_t conns;
    auto obj2conn = zip_view<Obj2Conn>(Obj2Conn(_objects, conns));
    
    for_each(obj2conn, detail::connect_object
             <From, Transform, remove_reference_t<F>, after_change_connection_t>
             {_objects, _transform, f});
    
    return {std::move(conns)};
  }

  /* ... */
private:
  mutable ranges::semiregular_t<Transform> _transform;
  From _objects;
};
\end{lstlisting}
\end{frame}

\begin{frame}[fragile,t]{Problema \#1 - Solução Coruja}
\lstset{basicstyle=\tiny\sffamily}
\begin{lstlisting}[
    linebackgroundcolor={%
      \btLstHL<1>{16-17}%
      \btLstHL<2>{23-25}%
      \btLstHL<3>{1-9}%
  }]
template<typename Objects, typename Transform, typename F, typename Observed>
struct lift_f : private Transform {
    /* ... */
    void operator()(const Observed&)
    { _f(boost::fusion::invoke(static_cast<Transform&>(*this), _objects)); }
    
    Objects _objects;
    F _f;
};
        
template<typename From, typename Transform, typename F, typename Conn>
struct connect_object {
    template<typename Obj2Conn>
    void operator()(Obj2Conn&& obj2conn) const {
        using namespace boost::fusion;
        auto& object = at_c<0>(obj2conn);
        auto& conn = at_c<1>(obj2conn);

        using Obj = typename std::remove_reference<
            typename result_of::at_c<Obj2Conn, 0>::type>::type;
        
        //TODO: Update from
        conn = object.after_change
            (lift_f<From, Transform, F, typename Obj::observed_t>
             (_transform, _from, _f));
    }
    
    From& _from;
    Transform& _transform;
    F& _f;
};
\end{lstlisting}
\end{frame}

\begin{frame}[fragile,t]{Performance AdHoc x Lift - assignment}
\lstset{basicstyle=\scriptsize\sffamily}
\begin{lstlisting}[
    emph={array,connection_t,conns,thickness,marker,print_thickness,print_new_thickness},
    % emphstyle={\color{hsrmRedDark}},
    emphstyle={\color{hsrmSec3CompDark}},
    gobble=0,
    linebackgroundcolor={%
      \btLstHL<1>{7,12}%
  }]
object<string> s1, s2;

//AdHoc
auto l = [&r,&o1,&o2](string){r = o1.observed() + o2.observed();};
o1.after_change(l);
o2.after_change(l);
o1 = "abc";

//Lift
auto s1s2 = s1 + s2;
s1s2.after_change([&r](string s){ r = s; });
o1 = "abc";
\end{lstlisting}
\begin{table}[]
  \footnotesize
  \begin{tabularx}{\linewidth}{X X X X}
    \toprule
    \textbf{Método} & \textbf{Média} & \textbf{Mediana} & \textbf{Desvio padrão} \tabularnewline 
    \midrule
    \texttt{lift} & \texttt{33ns} & \texttt{33ns} & \texttt{0ns} \tabularnewline
    \texttt{adHoc} & \texttt{33ns} & \texttt{33ns} & \texttt{1ns} \tabularnewline
    \bottomrule
  \end{tabularx}
  \label{tab:options}
\tiny AMD Ryzen 7 1700X with GCC 8.2.0 -O3 and 10x repetitions
\end{table}
\end{frame}

\begin{frame}[fragile,t]{Performance AdHoc x Lift - after\_change()}
\lstset{basicstyle=\scriptsize\sffamily}
\begin{lstlisting}[
    emph={array,connection_t,conns,thickness,marker,print_thickness,print_new_thickness},
    % emphstyle={\color{hsrmRedDark}},
    emphstyle={\color{hsrmSec3CompDark}},
    gobble=0,
    linebackgroundcolor={%
      \btLstHL<1>{3-5,10,11}%
  }]
//AdHoc
for(size_t i(0);i<state.range(0);++i) {
  auto c1 = o1.after_change(l);
  auto c2 = o2.after_change(l);
  conns.push_back({move(c1), move(c2)});
}

//Lift
for(size_t i(0);i<state.range(0);++i) {
  auto c = concat.after_change([&r](string s){r = s;});
  conns.push_back(move(c));
}
\end{lstlisting}
\newcolumntype{R}[1]{>{\raggedleft\let\newline\\\arraybackslash\hspace{0pt}}m{#1}}
\begin{table}[]
  \footnotesize
  \begin{tabularx}{\linewidth}{R{1.0cm} R{2.0cm} R{1.5cm} R{1.2cm} R{1.0cm} R{1.0cm}}
    \toprule
    \textbf{n} & \textbf{Lift($\mu$s)}  & \textbf{AdHoc($\mu$s)} & \textbf{Diff} & \textbf{sd($\mu$s)} & \textbf{sd($\mu$s)} \tabularnewline 
    \midrule
    \texttt{1} & \texttt{0,101} & \texttt{0,102} & \texttt{+0,01} & \texttt{0} & \texttt{0} \tabularnewline
    \texttt{10} & \texttt{1,032} & \texttt{1,033} & \texttt{+0,01} & \texttt{0,001} & \texttt{0,026} \tabularnewline
    \texttt{100} & \texttt{14,167} & \texttt{14,925} & \texttt{+0,05} & \texttt{0,006} & \texttt{0,101} \tabularnewline
    \bottomrule
  \end{tabularx}
  \label{tab:options}
\tiny AMD Ryzen 7 1700X with GCC 8.2.0 -O3 and 5x repetitions
\end{table}
\end{frame}

\section{Ranges observáveis}

\begin{frame}[fragile,t]{Problema \#2 - Visão em árvore}
\lstset{basicstyle=\footnotesize\sffamily}
\begin{lstlisting}[
    emph={marker,well,markers_t,flat_set,string},
    % emphstyle={\color{hsrmRedDark}},
    emphstyle={\color{hsrmSec3CompDark}},
    gobble=0,
    linebackgroundcolor={%
      \btLstHL<1>{1-3}%
      \btLstHL<2>{5-8}%
      \btLstHL<3>{10}%
    }]
struct city {
    string name;
};

struct country {
    string name;
    vector<city> cities;
};

vector<country> countries;
\end{lstlisting}
\end{frame}

\begin{frame}{Problema \#2 - Visão em árvore}
\begin{tikzpicture}[scale=0.6]
  \tikzset{every child/.append style={scale=1}}
  \path[mindmap,concept color=hsrmSec3CompDark,text=white]
  node[concept,scale=0.4] {Countries}
  [clockwise from=30]
  child[concept color=hsrmSec2Dark,text=white] { 
    node[concept, scale=0.5] {Country 1} 
    child { node[concept,color=hsrmSec1CompDark,text=white,scale=0.5] {City 1} } 
    child { node[concept,color=hsrmSec1CompDark,text=white,scale=0.5] {City 2} } 
    child { node[concept,color=hsrmSec1CompDark,text=white,scale=0.5] {City 3} } 
  }
  child[concept color=hsrmSec2Dark,text=white] { 
    node[concept, scale=0.5] {Country 2} 
    child { node[concept,color=hsrmSec1CompDark,text=white,,scale=0.5] {City A} } 
    child { node[concept,color=hsrmSec1CompDark,text=white,scale=0.5] {City B} } 
  };
\end{tikzpicture}
\begin{itemize}
\item Visão atualizada imediatamente quando o modelo é alterado
  \begin{itemize}
  \item Inserção/Remoção de nó
  \item Alteração do nome
  \end{itemize}
\end{itemize}
\end{frame}

\begin{frame}[fragile,t]{Problema \#2 - Solução Ad Hoc - Country}
\lstset{basicstyle=\scriptsize\sffamily}
\begin{lstlisting}[
    gobble=0,
    linebackgroundcolor={%
      \btLstHL<1>{15-16}%
      \btLstHL<2>{4-7,9,11-13}%
    }]
struct country {
  using cities_t = vector<city>;

  void push_back_city(const city& c) {
      _cities.push_back(c);
      _city_insert(_cities, prev(_cities.end()));
  }

  const cities_t& cities() const noexcept{ return _cities; };

  template<typename F>
  any_connection on_city_insert(F&& f)
  { return _city_insert.connect(forward<F>(f)); }
private:
  signal<void(const cities_t&, cities_t::iterator)>
    _city_insert;
  cities_t _cities;
};
\end{lstlisting}
\footnotesize
\only<2>{
\texttt{push\_back(city\&\&)}, \texttt{insert}, \texttt{emplace}, \texttt{emplace\_back} e \texttt{std::sort()}?
}
\end{frame}

\begin{frame}[fragile,t]{Problema \#2 - Solução Ad Hoc - Country}
\lstset{basicstyle=\tiny\sffamily}
\begin{lstlisting}[
    gobble=0,
    linebackgroundcolor={%
      \btLstHL<1->{10-13,21-23,25}%
    }]
struct country {
  string name;
  using cities_t = vector<city>;
  
  void push_back_city(const city& c) {
      _cities.push_back(c);
      _city_insert(_cities, prev(_cities.end()));
  }

  cities_t::iterator erase(cities_t::iterator it) {
      _city_erase(_cities, it);
      return _cities.erase(it);
  }
  
  const cities_t& cities() const noexcept { return _cities; };

  template<typename F>
  any_connection on_city_insert(F&& f)
  { return _city_insert.connect(forward<F>(f)); }

  template<typename F>
  any_connection on_city_erase(F&& f)
  { return _city_erase.connect(forward<F>(f)); }
private:
  signal<void(cities_t&,cities_t::iterator)> _city_insert, _city_erase;
  cities_t _cities;
};
\end{lstlisting}
\end{frame}

\begin{frame}[fragile,t]{Problema \#2 - Solução Ad Hoc - Countries}
\lstset{basicstyle=\tiny\sffamily}
\begin{lstlisting}
struct countries {
  using model_t = vector<country>;
  
  void push_back_country(const country& c) {
      _model.push_back(c);
      _country_insert(_model, prev(_model.end()));
  }

  model_t::iterator erase(model_t::iterator it) {
      _country_erase(_model, it);
      return _model.erase(it);
  }
  
  const model_t& countries() const noexcept { return _model; };

  template<typename F>
  any_connection on_country_insert(F&& f)
  { return _country_insert.connect(forward<F>(f)); }

  template<typename F>
  any_connection on_country_erase(F&& f)
  { return _country_erase.connect(forward<F>(f)); }
private:
  signal<void(model_t&, model_t::iterator)>
    _country_insert, _country_erase;
  model_t _model;
};
\end{lstlisting}
\end{frame}

\begin{frame}[fragile,t]{Problema \#2 - Solução Ad Hoc - Observação}
\lstset{basicstyle=\scriptsize\sffamily}
\begin{lstlisting}[
    emph={basin,it,wells,well,markers,marker},
    % emphstyle={\color{hsrmRedDark}},
    emphstyle={\color{hsrmSec3CompDark}},
    gobble=0,
    linebackgroundcolor={%
      \btLstHL<1>{3-4}%
      \btLstHL<2>{5}%
      \btLstHL<3>{6}%
      \btLstHL<4>{8-9}%
      \btLstHL<5>{10-11}%
      \btLstHL<6>{14}%
  }]
std::vector<any_connection> conns;  
conns.push_back(
countries.on_country_insert(
  [&conns,st](auto& countries, auto it) {
    auto pos = distance(countries.begin(),it);
    auto gtk_it = insert_row(st, it->name, to_string(pos));
    conns.push_back(
      it->on_city_insert.connect(
      [gtk_it,st](auto& cities, auto it) {
        auto pos = distance(cities.begin(),it);
        insert_row_child(st, it->name, gtk_it, pos);
      }));
  }));
for(auto& c : conns) c.disconnect();  
\end{lstlisting}
\end{frame}

\begin{frame}[fragile,t]{Problema \#2 - Solução Ad Hoc - Observação}
\lstset{basicstyle=\scriptsize\sffamily}
\begin{lstlisting}[
    emph={basin,it,wells,well,markers,marker},
    % emphstyle={\color{hsrmRedDark}},
    emphstyle={\color{hsrmSec3CompDark}},
    gobble=0,
    linebackgroundcolor={%
      \btLstHL<1>{7-9,15-17}%
    }]
std::vector<any_connection> conns;  
conns.push_back(
countries.on_country_insert(
  [&conns,st](auto& countries, auto it) {
    auto pos = distance(countries.begin(),it);
    auto gtk_it = insert_row(st, it->name, to_string(pos));
    conns.push_back(
    it->after_change(
      [gtk_it,st](string s){update_row(st, s, gtk_it);}));
    conns.push_back(
      it->on_city_insert.connect(
      [gtk_it,st, &conns](auto& cities, auto it) {
        auto pos = distance(cities.begin(),it);
        auto gtk_it = insert_row_child(st, it->name, gtk_it, pos);
        conns.push_back(
        it->after_change(
          [gtk_it,st](string s){update_row(st, s, gtk_it);}));
        }));
  }));
for(auto& c : conns) c.disconnect();  
\end{lstlisting}
\end{frame}

\begin{frame}[fragile,t]{Problema \#2 - Solução Ad Hoc - Observação}
\lstset{basicstyle=\tiny\sffamily}
\begin{lstlisting}[
    emph={basin,it,wells,well,markers,marker},
    % emphstyle={\color{hsrmRedDark}},
    emphstyle={\color{hsrmSec3CompDark}},
    gobble=0,
    linebackgroundcolor={%
      \btLstHL<1>{19-24}%
    }]
std::vector<any_connection> conns;  
conns.push_back(
countries.on_country_insert(
  [&conns,st](auto& countries, auto it) {
    auto pos = distance(countries.begin(),it);
    auto gtk_it = insert_row(st, it->name, to_string(pos));
    conns.push_back(
      it->after_change(
      [gtk_it,st](string s){update_row(st, s, gtk_it);}));
    conns.push_back(
      it->on_city_insert.connect(
      [gtk_it,st,&conns](auto& cities, auto it) {
        auto pos = distance(cities.begin(),it);
        auto gtk_it = insert_row_child(st, it->name, gtk_it, pos);
        conns.push_back(
          it->after_change(
          [gtk_it,st](string s){update_row(st, s, gtk_it);}));
      }));
    conns.push_back(
      it->on_city_erase.connect(
        [&conns,st](auto& cities, auto it) {
          auto pos = distance(cities.begin(),it);
          remove_row(st,to_string(pos));
        }));
  }));

for(auto& c : conns) c.disconnect();  
\end{lstlisting}
\end{frame}

\begin{frame}[fragile,t]{Problema \#2 - Solução Ad Hoc - Observação}
\lstset{basicstyle=\scriptsize\sffamily}
\begin{lstlisting}[
    emph={basin,it,wells,well,markers,marker},
    % emphstyle={\color{hsrmRedDark}},
    emphstyle={\color{hsrmSec3CompDark}},
    gobble=0,
    linebackgroundcolor={%
    }]
conns.push_back(
countries.on_country_erase(
  [&conns,st](auto& countries, auto it) {
    auto pos = distance(countries.begin(),it);
    remove_row(st,to_string(pos));
  }));
\end{lstlisting}
\end{frame}

\begin{frame}[fragile,t]{Problema \#2 - Solução Coruja}
\lstset{basicstyle=\footnotesize\sffamily}
\begin{lstlisting}[
    gobble=0,
    linebackgroundcolor={%
      \btLstHL<1>{7,10}%
    }]
struct city {
    string name;
};

struct country {
    string name;
    coruja::vector<city> cities;
};

coruja::vector<country> countries;
\end{lstlisting}
\end{frame}

\begin{frame}[fragile,t]{Ranges}
\begingroup
\setbeamercolor{block title}{bg=hsrmSec2Dark}
\setbeamercolor{block body}{bg=hsrmSec2}
\begin{block}{Range}
Representa uma sequência de elementos através de [begin(), end())
\end{block}
\endgroup
\lstset{
basicstyle=\footnotesize\sffamily,
numbers=none
}
\begin{onlyenv}<1>
\begin{lstlisting}
countries | transform([](auto& c){return c.name;});
//{"Country", "Country2"}
\end{lstlisting}
\end{onlyenv}
\end{frame}

\begin{frame}[fragile,t]{Problema \#2 - Solução Coruja}
\lstset{basicstyle=\footnotesize\sffamily}
\begingroup
\setbeamercolor{block title}{bg=hsrmSec2Dark}
\setbeamercolor{block body}{bg=hsrmSec2}
\begin{block}{ObservableErasableRange}
\emph{Observable} e \emph{Range} permitindo a observação na inserção e remoção de elementos.
\end{block}
\endgroup
\lstset{
basicstyle=\scriptsize\sffamily,
numbers=none
}
\begin{table}[]
  \scriptsize
  \begin{tabularx}{\linewidth}{l>{\raggedright}X}
    \toprule
    \textbf{Requisitos}   \tabularnewline 
    \midrule
    \begin{lstlisting}^^J
      //FunctionObject: void(Rng\&, Rng::iterator)^^J
      //          void(reference_t<Rng>)^^J
      template<typename FunctionObject>^^J
      for_each_connection_t for_each(FunctionObject)^^J
    \end{lstlisting}
    \tabularnewline
    \tabularnewline
    \begin{lstlisting}^^J
      //FunctionObject: void(Rng\&, Rng::iterator)^^J
      //          void(reference_t<Rng>)^^J
      template<typename FunctionObject>^^J
      before_erase_connection_t before_erase(FunctionObject)^^J
    \end{lstlisting}
    \tabularnewline
    \tabularnewline
    \begin{lstlisting}^^J
      Observable::for_each_connection_t^^J
      Observable::before_erase_connection_t^^J
    \end{lstlisting}
    \tabularnewline
    \bottomrule
  \end{tabularx}
  \label{tab:options}
\end{table}
\end{frame}

\begin{frame}[fragile,t]{Problema \#2 - Solução Coruja}
\lstset{basicstyle=\scriptsize\sffamily}
\begin{lstlisting}[
    emph={basin,it,wells,well,markers,marker},
    % emphstyle={\color{hsrmRedDark}},
    emphstyle={\color{hsrmSec3CompDark}},
    gobble=0,
    linebackgroundcolor={%
      \btLstHL<1>{1,2}%
      \btLstHL<2>{4}%
    }]
template<typename ObservableErasableRange, typename F>
auto transform(ObservableErasableRange&&, F&&)

transform(countries, [](country& c){return c.name;});
//{"Country", "Country2"}
\end{lstlisting}
\end{frame}

\begin{frame}[fragile,t]{Problema \#2 - Solução Coruja}
\lstset{basicstyle=\footnotesize\sffamily}
\begin{lstlisting}[
    emph={basin,it,wells,well,markers,marker},
    % emphstyle={\color{hsrmRedDark}},
    emphstyle={\color{hsrmSec3CompDark}},
    gobble=0,
    linebackgroundcolor={%
      \btLstHL<1>{1-9}%
      \btLstHL<2>{11-12}%
      \btLstHL<3>{14-17}%
  }]
//Rows is an ObservableErasableRange
//Rows::value_type should be Row:
//Row : string
//    | pair<string, Rows>
template<typename Rows>
struct tree_t {
    explicit tree_t(Rows) {/*impl*/}
    Rows rows;
};

//{("Country1", {"City1","City2","City3"}),
// ("Country2", {"CityA","CityB"})}

coruja::vector<string> v{"abc", "def"};
auto tree = make_tree(v);
v.emplace_back("ghi"); //update tree
\end{lstlisting}
\end{frame}

\begin{frame}[fragile,t]{Problema \#2 - Solução Coruja}
\lstset{basicstyle=\footnotesize\sffamily}
\begin{lstlisting}[
    emph={basin,it,wells,well,markers,marker},
    % emphstyle={\color{hsrmRedDark}},
    emphstyle={\color{hsrmSec3CompDark}},
    gobble=0,
    linebackgroundcolor={%
      \btLstHL<1>{11}%
      \btLstHL<2>{14}%
  }]
//Rows is an ObservableErasableRange
//Rows::value_type should be Row:
//Row : string
//    | pair<string, Rows>
template<typename Rows>
struct tree_t {
    explicit tree_t(Rows) {/*impl*/}
    Rows rows;
};

coruja::vector<coruja::object<string>> v;
auto tree = make_tree(v);
v.emplace_back("ghi"); //update tree
v.back() = "change"; //update tree
\end{lstlisting}
\end{frame}

\begin{frame}[fragile,t]{Problema \#2 - Solução Coruja}
\lstset{basicstyle=\footnotesize\sffamily}
\begin{lstlisting}[
    emph={basin,it,wells,well,markers,marker},
    % emphstyle={\color{hsrmRedDark}},
    emphstyle={\color{hsrmSec3CompDark}},
    gobble=0,
    linebackgroundcolor={%
      \btLstHL<1>{8,9}%
      \btLstHL<2>{10-11}%
  }]
struct city { string name; };
struct country {
    string name;
    coruja::vector<city> cities;
};
coruja::vector<country> countries;

auto rows = transform(countries, 
  [](country& c){
    return row(c.name, transform(c.cities,
               [](city& c){return c.name;})); });
  
auto tree = make_tree(rows);
//{("Country1", {"City1","City2","City3"}),
// ("Country2", {"CityA","CityB"})}
\end{lstlisting}
\end{frame}

\begin{frame}[fragile,t]{Problema \#2 - Solução Coruja}
\lstset{basicstyle=\tiny\sffamily}
\begin{lstlisting}[
    emph={basin,it,wells,well,markers,marker},
    % emphstyle={\color{hsrmRedDark}},
    emphstyle={\color{hsrmSec3CompDark}},
    gobble=0,
    linebackgroundcolor={%
      \btLstHL<1>{6-7}%
      \btLstHL<2>{9-10}%
  }]
template<typename F>
class invoke_observer_impl : private invoke_observer_base<F> {
  using base = invoke_observer_base<F>;
  using base::base;
  
  template<typename From, typename It>
  void operator()(From& from, It it) {
    using namespace ranges;
    auto rng = coruja_transform_view{from, base::as_transform()};
    base::_f(rng, next(begin(rng), distance(begin(from), it)));
  }
};
\end{lstlisting}
\end{frame}

\begin{frame}[fragile,t]{push\_back() - AdHoc x Transform}
  \newcolumntype{R}[1]{>{\raggedleft\let\newline\\\arraybackslash\hspace{0pt}}m{#1}}
  \begin{table}[]
  \scriptsize
  \begin{tabularx}{\linewidth}{R{1.7cm} R{2.0cm} R{1.5cm} R{1.2cm} R{1.0cm} R{1.0cm}}
    \toprule
    \textbf{push\_back()} & \textbf{Transform($\mu$s)}  & \textbf{AdHoc($\mu$s)} & \textbf{Diff($\mu$s)} & \textbf{sd($\mu$s)} & \textbf{sd($\mu$s)} \tabularnewline 
    \midrule
    \texttt{50} & \texttt{83} & \texttt{83} & \texttt{0} & \texttt{4} & \texttt{4} \tabularnewline
    \texttt{200} & \texttt{435} & \texttt{423} & \texttt{-12} & \texttt{20} & \texttt{20} \tabularnewline
    \texttt{500} & \texttt{2000} & \texttt{2019} & \texttt{+19} & \texttt{24} & \texttt{32} \tabularnewline
    \texttt{1000} & \texttt{7398} & \texttt{7476} & \texttt{+78} & \texttt{203} & \texttt{249} \tabularnewline
    \texttt{1500} & \texttt{16206} & \texttt{16301} & \texttt{+95} & \texttt{352} & \texttt{389} \tabularnewline
    \texttt{2000} & \texttt{28043} & \texttt{28310} & \texttt{+267} & \texttt{326} & \texttt{441} \tabularnewline
    \texttt{5000} & \texttt{188646} & \texttt{191184} & \texttt{+2538} & \texttt{2408} & \texttt{578} \tabularnewline
    \texttt{10000} & \texttt{841242} & \texttt{854977} & \texttt{+13735} & \texttt{10371} & \texttt{9148} \tabularnewline
    \bottomrule
  \end{tabularx}
  \label{tab:options}
\tiny AMD Ryzen 7 1700X with GCC 4.8.2 -O3, GTK 2.24/GtkTreeStore and 10-250x repetitions
\end{table}
\end{frame}

\begin{frame}[fragile,t]{push\_back() - std::vector x Coruja x Boost.Signals2}
\lstset{basicstyle=\footnotesize\sffamily}
\begin{lstlisting}[
    emph={basin,it,wells,well,markers,marker},
    % emphstyle={\color{hsrmRedDark}},
    emphstyle={\color{hsrmSec3CompDark}},
    gobble=0,
    linebackgroundcolor={%
  }]
//wosignals  
using vec_t = std::vector<size_t>;

//csignals(Coruja signal)
using vec_t = vector<size_t>;

//bsignals(Boost.Signals2)
using vec_t = vector<size_t, allocator<size_t>,
                     std::vector,void,boost_signals2>;

\end{lstlisting}

  \newcolumntype{R}[1]{>{\raggedleft\let\newline\\\arraybackslash\hspace{0pt}}m{#1}}
  \begin{table}[]
  \scriptsize
  \begin{tabularx}{\linewidth}{R{0.8cm} R{1.2cm} R{1.3cm} R{1.4cm} R{0.9cm} R{0.8cm} R{0.8cm} R{1.1cm}}
    \toprule
    \textbf{n} & \textbf{vector(ns)}  & \textbf{coruja(ns)} & \textbf{boost(ns)} & \textbf{diff} & \textbf{sd(ns)} & \textbf{sd(ns)} & \textbf{sd(ns)} \tabularnewline 
    \midrule
    \texttt{5} & \texttt{26,84} & \texttt{28,76} & \texttt{223,96} & \texttt{+6,79} & \texttt{0,10} & \texttt{0,32} & \texttt{14,36} \tabularnewline
    \texttt{1000} & \texttt{2083,38} & \texttt{2160,98} & \texttt{38497,10} & \texttt{+16,81} & \texttt{1,38} & \texttt{0,37} & \texttt{48,37} \tabularnewline
    \texttt{10000} & \texttt{20594,15} & \texttt{21388,84} & \texttt{388575,65} & \texttt{+17,17} & \texttt{0,78} & \texttt{6,27} & \texttt{3929,02} \tabularnewline
    \bottomrule
  \end{tabularx}
  \label{tab:options}
\tiny AMD Ryzen 7 1700X with GCC 8.2.0 -O3, Boost 1.67.0 and 5 repetitions
\end{table}
\end{frame}

\begin{frame}[fragile,t]{signal - Coruja x Boost.Signals2 - compile time}
\lstset{basicstyle=\scriptsize\sffamily}
\begin{lstlisting}[
    emph={basin,it,wells,well,markers,marker},
    % emphstyle={\color{hsrmRedDark}},
    emphstyle={\color{hsrmSec3CompDark}},
    gobble=0,
    linebackgroundcolor={%
  }]
//Coruja signal
#include <coruja/support/signal.hpp>
coruja::signal<void(int)> sig;

//Boost.Signals2
#include <boost/signals2/signal.hpp>
boost::signals2::signal<void(int)> sig;
\end{lstlisting}

  \newcolumntype{R}[1]{>{\raggedleft\let\newline\\\arraybackslash\hspace{0pt}}m{#1}}
  \begin{table}[]
  \scriptsize
  \begin{tabularx}{\linewidth}{R{1.5cm} R{3.0cm} R{1.2cm} R{1.2cm} R{1.2cm}}
    \toprule
    \textbf{Coruja(s)} & \textbf{Boost.Signals2(s)} & \textbf{Diff} & \textbf{sd(s)} & \textbf{sd(s)} \tabularnewline 
    \midrule
    \texttt{0.286} & \texttt{1.298} & \texttt{+3.54} & \texttt{0.005} & \texttt{0.004} \tabularnewline
    \bottomrule
  \end{tabularx}
  \label{tab:options}
\tiny AMD Ryzen 7 1700X with GCC 8.2.0 -O3, Boost 1.67.0 and 5 repetitions
\end{table}
\end{frame}

\begin{frame}[fragile,t]{Outros modelos}
  \begin{itemize}
    \item coruja::map, coruja::unordered\_map e coruja::flat\_map
    \item coruja::set e coruja::flat\_set
    \item coruja::optional
    \item coruja::variant
    \item coruja::object\_view e coruja::container\_view
    \item coruja::any\_object\_view
  \end{itemize}
\end{frame}

\begin{frame}[fragile,t]{Suporte Boost.Serialization}
\lstset{basicstyle=\footnotesize\sffamily}
\begin{lstlisting}[
    emph={basin,it,wells,well,markers,marker},
    % emphstyle={\color{hsrmRedDark}},
    emphstyle={\color{hsrmSec3CompDark}},
    gobble=0,
    linebackgroundcolor={%
      \btLstHL<1>{4}%
  }]
template<typename Archive>
void serialize(Archive& ar, country& o, unsigned int) {
  ar & o.name;
  ar & o.cities; //coruja::vector<city>
}
\end{lstlisting}
\begin{itemize}
\item Operação de \texttt{load} notifica observadores
\end{itemize}
\end{frame}

\begin{frame}[fragile,t]{Trabalhos relacionados}
  \begin{itemize}
    \item RxCpp
      \begin{itemize}
        \item Abstrações baseadas em FRP (streams e operations)
        \item Notificações assíncronas
        \item `rx::iterate` (Observables a partir de ranges)
          \begin{itemize}
            \item Considera o range como imutável.
          \end{itemize}
      \end{itemize}
    \item Sodium e sfrp
      \begin{itemize}
        \item Implementações puras de FRP (stream, cell e operations)
      \end{itemize}
  \end{itemize}
\end{frame}

\begin{frame}[fragile,t]{Continuação}
  \begin{itemize}
    \item Revisitar classes reativas: \texttt{coruja::reactive\_class}
    \item Lançar versão 0.1
    \item Revisitar suporte a FRP (streams e operations)
    \item Melhorar o uso de concepts em compile time
    \item Explorar widgets com suporte a Observables
  \end{itemize}
\end{frame}

\begin{frame}[fragile,t]{Obrigado}
  \href{https://github.com/ricardocosme/coruja}{github.com/ricardocosme/coruja}
  \begin{itemize}
    \item Benchmarks: \texttt{test/bench*.cpp}
  \end{itemize}
\end{frame}
\end{document}
