%!TEX program = xelatex

% Name           : hsrm-beamer-demo.sty
% Author         : Benjamin Weiss (benjamin.weiss@kreatiefton.de)
% Version        : 0.4
% Created on     : 05.05.2013
% Last Edited on : 24.03.2014
% Copyright      : Copyright (c) 2013-2014 by Benjamin Weiss. All rights reserved.
% License        : This file may be distributed and/or modified under the
%                  GNU Public License.
% Description    : HSRM beamer theme demonstration. Also includes a short 
%                  Tutorial regarding the beamer class.

\documentclass[compress]{beamer}
%--------------------------------------------------------------------------
% Common packages
%--------------------------------------------------------------------------
\usepackage[portuguese]{babel}
\usepackage{pgfplots}
\usepackage{array}
\usepackage{graphicx}
\usepackage{multicol}
% Erweiterte Tabellenfunktionen
\usepackage{tabularx,ragged2e}
\usepackage{booktabs}
% Listingserweiterung
\usepackage{pgf, pgffor}
\usepackage{listings}
\usepackage{dirtree}

%% \usepackage{lstlinebgrd}

%% hack from https://tex.stackexchange.com/questions/451532/recent-issues-with-lstlinebgrd-package-with-listings-after-the-latters-updates
\makeatletter
\let\old@lstKV@SwitchCases\lstKV@SwitchCases
\def\lstKV@SwitchCases#1#2#3{}
\makeatother
\usepackage{lstlinebgrd}
\makeatletter
\let\lstKV@SwitchCases\old@lstKV@SwitchCases

\lst@Key{numbers}{none}{%
    \def\lst@PlaceNumber{\lst@linebgrd}%
    \lstKV@SwitchCases{#1}%
    {none:\\%
     left:\def\lst@PlaceNumber{\llap{\normalfont
                \lst@numberstyle{\thelstnumber}\kern\lst@numbersep}\lst@linebgrd}\\%
     right:\def\lst@PlaceNumber{\rlap{\normalfont
                \kern\linewidth \kern\lst@numbersep
                \lst@numberstyle{\thelstnumber}}\lst@linebgrd}%
    }{\PackageError{Listings}{Numbers #1 unknown}\@ehc}}
\makeatother

\usepackage{hyperref}
\usepackage{fontspec}
\usepackage{siunitx}
\sisetup{math-micro=\text{µ},text-micro=µ}

\makeatletter
%%%%%%%%%%%%%%%%%%%%%%%%%%%%%%%%%%%%%%%%%%%%%%%%%%%%%%%%%%%%%%%%%%%%%%%%%%%%%%
%
% \btIfInRange{number}{range list}{TRUE}{FALSE}
%
% Test in int number <number> is element of a (comma separated) list of ranges
% (such as: {1,3-5,7,10-12,14}) and processes <TRUE> or <FALSE> respectively

\newcount\bt@rangea
\newcount\bt@rangeb

\newcommand\btIfInRange[2]{%
    \global\let\bt@inrange\@secondoftwo%
    \edef\bt@rangelist{#2}%
    \foreach \range in \bt@rangelist {%
        \afterassignment\bt@getrangeb%
        \bt@rangea=0\range\relax%
        \pgfmathtruncatemacro\result{ ( #1 >= \bt@rangea) && (#1 <= \bt@rangeb) }%
        \ifnum\result=1\relax%
            \breakforeach%
            \global\let\bt@inrange\@firstoftwo%
        \fi%
    }%
    \bt@inrange%
}
\newcommand\bt@getrangeb{%
    \@ifnextchar\relax%
        {\bt@rangeb=\bt@rangea}%
        {\@getrangeb}%
}
\def\@getrangeb-#1\relax{%
    \ifx\relax#1\relax%
        \bt@rangeb=100000%   \maxdimen is too large for pgfmath
    \else%
        \bt@rangeb=#1\relax%
    \fi%
}

%%%%%%%%%%%%%%%%%%%%%%%%%%%%%%%%%%%%%%%%%%%%%%%%%%%%%%%%%%%%%%%%%%%%%%%%%%%%%%
%
% \btLstHL<overlay spec>{range list}
%
% TODO BUG: \btLstHL commands can not yet be accumulated if more than one overlay spec match.
% 
\newcommand<>{\btLstHL}[1]{%
  \only#2{\btIfInRange{\value{lstnumber}}{#1}{\color{yellow!20}\def\lst@linebgrdcmd{\color@block}}{\def\lst@linebgrdcmd####1####2####3{}}}%
}%
\makeatother

\lstdefinestyle{emph}{
  % keywordstyle=\color{orange}\bfseries
  linebackgroundcolor=\color{orange!15}
}

\lstset{
literate={~} {$\sim$}{1}, % set tilde as a literal (no process)
escapeinside=@@,
language=c++,
% basicstyle=\small,
basicstyle=\normalsize\sffamily,
keywordstyle=\color{hsrmSec1Dark}\bfseries,
stringstyle=\color{red},
commentstyle=\color{gray},
morecomment=[l][\color{magenta}]{\#},
numbers=left,
numberstyle=\tiny\ttfamily,
stepnumber=1,
showspaces=false,
showstringspaces=false,
showtabs=false,
breaklines=true,
% frame=tb,
framerule=0.5pt,
tabsize=2,
framexleftmargin=0.5em,
framexrightmargin=0.5em,
xleftmargin=0.5em,
xrightmargin=0.5em,
}

%--------------------------------------------------------------------------
% Load theme
%--------------------------------------------------------------------------
\usetheme{hsrm}

% \usepackage{dtklogos} % must be loaded after theme
\usepackage{tikz}
\usetikzlibrary{mindmap,backgrounds}

%--------------------------------------------------------------------------
% General presentation settings
%--------------------------------------------------------------------------
\title{Widgets da Saci}
\subtitle{Usando \texttt{coruja::object<T>} como modelos em widgets Qt}
\date{\today}
\author{Ricardo Cosme}
\institute{Instituto Tecgraf de Desenvolvimento de \\ Software Técnico-Científico da PUC-Rio\\{\Medium Tecgraf/PUC-Rio}}

%--------------------------------------------------------------------------
% Notes settings
%--------------------------------------------------------------------------
\setbeameroption{show notes}

\begin{document}
%--------------------------------------------------------------------------
% Titlepage
%--------------------------------------------------------------------------

\maketitle

%\begin{frame}[plain]
%	\titlepage
%\end{frame}

%--------------------------------------------------------------------------
% Table of contents
%--------------------------------------------------------------------------
% \section*{Seções}
% \begin{frame}{Seções}
% 	% hideallsubsections ist empfehlenswert für längere Präsentationen
% 	\tableofcontents[hideallsubsections]
% \end{frame}

%--------------------------------------------------------------------------
% Content
%--------------------------------------------------------------------------
\section{Widgets Qt}

\begin{frame}[fragile,t]{Solução Adhoc \#1 - Modelo não observável}
\lstset{basicstyle=\scriptsize\sffamily}
\begin{lstlisting}[
    gobble=0,
    linebackgroundcolor={%
      \btLstHL<1>{1}%
      \btLstHL<2>{3}%
      \btLstHL<3>{11-13}%
      \btLstHL<4>{15-16,7-8}%
  }]
struct person_t { double height; };

class view_t : public QObject {
  Q_OBJECT
public:
  view_t() : height(&window) {
    QObject::connect(&height, SIGNAL(valueChanged(double)),
                     this, SLOT(height_change_slot(double)));
  }
  virtual ~view_t() = default;
  QMainWindow window;
  QDoubleSpinBox height;
  coruja::signal<void(double)> on_height_change;
public Q_SLOTS:
  void height_change_slot(double v)
  { on_height_change(v); }
};

\end{lstlisting}
\end{frame}

\begin{frame}[fragile,t]{Solução Adhoc \#1 - Modelo não observável}
\lstset{basicstyle=\scriptsize\sffamily}
\begin{lstlisting}[
    gobble=0,
    linebackgroundcolor={%
      \btLstHL<1>{2}%
      \btLstHL<2->{5-6}%
  }]
//inicializa view
view.height.setValue(person.height);

//atualiza modelo
_conn = view.on_height_change.connect(
  [&](double v){ person.height = v; });
\end{lstlisting}
\only<3->{
  Modelo não observável
  \begin{itemize}
    \item<3->{E se o modelo tem múltiplas visões?}
    \item<4->{Solução: Padrão observador }
  \end{itemize}
}
\end{frame}

\begin{frame}[fragile,t]{Solução Adhoc \#2 - Modelo observável}
\lstset{basicstyle=\scriptsize\sffamily}
\begin{lstlisting}[
    gobble=0,
    linebackgroundcolor={%
      \btLstHL<1>{1}%
      \btLstHL<2->{3-4}%
  }]
struct person_t { coruja::object<double> height; };

conn_ = person.height.for_each(
  [&](double v){ view.height.setValue(v); });
\end{lstlisting}
\only<3->{
  Ciclo view -> model -> view
}
\end{frame}

\begin{frame}[fragile,t]{Solução Adhoc \#2 - Modelo observável}
\lstset{basicstyle=\scriptsize\sffamily}
\begin{lstlisting}[
    gobble=0,
    linebackgroundcolor={%
      \btLstHL<1>{1}%
      \btLstHL<2>{3-7}%
  }]
view.height_conn = conn_.get();

void view_t::height_change_slot(double v) {
  if(height_conn == coruja::any_connection{}) return;
  auto bc = make_scoped_blocked_connection(height_conn);
  on_height_change(v);
}
\end{lstlisting}
\end{frame}

\section{Widgets Saci}

\begin{frame}[fragile,t]{Solução Saci}
\lstset{basicstyle=\scriptsize\sffamily}
\begin{lstlisting}[
    gobble=0,
    linebackgroundcolor={%
      \btLstHL<1>{1}%
      \btLstHL<2>{4-8}%
  }]
struct person_t { coruja::object<double> height; };

struct view_t : window_t {
    view_t(coruja::object<double>& mheight)
    : height(mheight, child<QDoubleSpinBox>("height")) {}
    
    saci::qt::spinbox height;
};
\end{lstlisting}
\end{frame}

\begin{frame}{Instâncias}
  Classes não deveriam ser responsáveis pelo \textit{lifetime} de instâncias
  \begin{itemize}
  \item<1->Lógica estranha em C++
  \item<2->Hoje é {\Medium uma} instância, amanhã pode ser {\Medium N}
  \item<3->Por que sou obrigado a usar alocação dinâmica?
  \end{itemize}
\end{frame}

\begin{frame}{Mau uso}
  \begin{itemize}
  \item<1->Janelas filhas (Qual é a janela pai(dono)?)
  \item<2->Managers (Projeto como dono?)
  \end{itemize}
\end{frame}

\begin{frame}{Please!}
  \begin{itemize}
  \item<1-> Não use \textit{Singletons} indiscriminadamente
  \item<2-> Suícidio não é bom na vida real nem na programação
    \begin{itemize}
    \item<2-> \texttt{delete this;}
    \end{itemize}
  \end{itemize}
\end{frame}

\section{Janelas filha (Single)}

\begin{frame}[fragile,t]{Janela filha (Single)}
\begin{lstlisting}[
    gobble=0,
    linebackgroundcolor={%
      \btLstHL<1>{1-10}%
      \btLstHL<2>{12-13}%
    }]
class ToolXPresenter {
  ToolXPresenter() = default;
  static ToolXPresenter* _instance;
public:    
  static ToolXPresenter& instance() {
    if(!_instance)
      _instance = new ToolXPresenter();
    return *_instance;
  }
  void close() { delete this; }
};
void MainPresenter::openToolX() 
{ ToolXPresenter::instance().show(); }
\end{lstlisting}
\end{frame}

\begin{frame}[fragile,t]{Solução AdHoc}
\begin{lstlisting}[
    gobble=0,
    linebackgroundcolor={%
      \btLstHL<1>{2}%
      \btLstHL<2>{4-6}%
  }]
class ToolXPresenter {
  function<void()> _onClose;
public:    
  template<typename FunctionObject>
  void onClose(FunctionObject cbk)
  { _onClose = std::move(cbk); }
};
\end{lstlisting}
\end{frame}

\begin{frame}[fragile,t]{Solução AdHoc}
\begin{lstlisting}[
    gobble=0,
    linebackgroundcolor={%
      \btLstHL<1>{2}%
      \btLstHL<2>{6}%
      \btLstHL<3>{7-8}%
      \btLstHL<4>{9}%
      \btLstHL<5>{10}%
  }]
struct MainPresenter {
  unique_ptr<ToolXPresenter> toolx;
};

void MainPresenter::openToolX() {
  if(!toolx) {
    unique_ptr<ToolXPresenter> o
      (new ToolXPresenter(view));
    o->onClose([this]{ toolx.release(); });
    toolx = std::move(o);
  }
  toolx.show();
}
\end{lstlisting}
\end{frame}

\begin{frame}[fragile,t]{Solução AdHoc - Por valor}
\begin{lstlisting}[
    gobble=0,
    linebackgroundcolor={%
      \btLstHL<1>{3}%
      \btLstHL<2>{7}%
      \btLstHL<3>{8}%
      \btLstHL<4>{9-10}%
  }]
//MainPresenter models SemiRegular
struct MainPresenter {
  ToolXPresenter toolx;
};

void MainPresenter::openToolX() {
  if(toolx == ToolXPresenter()) {
    toolx = ToolXPresenter(view);
    toolx.onClose
      ([this]{ toolx = ToolXPresenter(); });
  }
  toolx->show();
}
\end{lstlisting}
\end{frame}

\section{Janelas filhas (Coleção)}

\begin{frame}[fragile,t]{Janelas filhas (Coleção)}
\begin{lstlisting}[
    gobble=0,
    linebackgroundcolor={%
      \btLstHL<1>{3-4}%
      \btLstHL<2>{8-9}%
    }]
struct ToolXPresenter {
  ToolXPresenter(view) {}
  //Não posso alocar ToolXPresenter na pilha
  void close() { delete this; }
};

void MainPresenter::openToolX() {
  //Memory leak?! Quem é o dono?
  auto toolx = new ToolXPresenter(view);
  toox->show();
}
\end{lstlisting}
\end{frame}

\begin{frame}[fragile,t]{Solução AdHoc}
\begin{lstlisting}[
    gobble=0,
    linebackgroundcolor={%
      \btLstHL<1>{2}%
      \btLstHL<2>{6-7}%
      \btLstHL<3>{8-9}%
      \btLstHL<4>{10}%
  }]
struct MainPresenter {
  list<unique_ptr<ToolXPresenter>> toolxs;
};

void MainPresenter::openToolX() {
  auto it = toolxs.emplace
    (toolxs.end(), new ToolXPresenter(view);
  it->get()->onClose
    ([this, it]{ toolxs.erase(it); });
  it->get()->show();
}
\end{lstlisting}
\end{frame}

\section{Solução genérica}

\begin{frame}[fragile,t]{Single - Alocação dinâmica - Implementação}
\lstset{basicstyle=\scriptsize\sffamily}
\begin{lstlisting}
template<typename Presenter>
class dyn_presenter {
  std::unique_ptr<Presenter> _presenter;
public:
  dyn_presenter() = default;
  
  Presenter* operator->() const noexcept
  { return _presenter.get(); }
  
  template<typename... Args>
  Presenter& instance(Args&&... args) {
    if(!_presenter) {
      std::unique_ptr<Presenter> o(
          new Presenter(std::forward<Args>(args)...));
      o->onClose([this]{ _presenter.release(); });
      _presenter = std::move(o);
    }
    return *_presenter;
  }
};
\end{lstlisting}
\end{frame}

\begin{frame}[fragile,t]{Single - Alocação dinâmica - Uso}
\begin{lstlisting}[
    gobble=0,
    linebackgroundcolor={%
      \btLstHL<1>{2}%
      \btLstHL<2>{6}%
  }]
struct MainPresenter {
  dyn_presenter<ToolXPresenter> toolx;
};

void MainPresenter::openToolX()
{ toolx.instance(view).show(); }
\end{lstlisting}
\end{frame}

\begin{frame}[fragile,t]{Single - Por valor - Uso}
\begin{lstlisting}[
    gobble=0,
    linebackgroundcolor={%
      \btLstHL<1>{6}%
  }]
struct MainPresenter {
  presenter<ToolXPresenter> toolx;
};

void MainPresenter::openToolX()
{ SACI_SINGLE_OF(toolx, view).show(); }
\end{lstlisting}
\end{frame}

\begin{frame}[fragile,t]{Coleção - Alocação dinâmica - Implementação}
\lstset{basicstyle=\scriptsize\sffamily}
\begin{lstlisting}[
    gobble=0,
    linebackgroundcolor={%
  }]
template<typename Presenter>
class dyn_presenters {
  std::list<std::unique_ptr<Presenter>> _presenters;
public:
  dyn_presenters() = default;
  
  template<typename... Args>
  Presenter& instance(Args&&... args) {
    auto it = _presenters.emplace
      (_presenters.end(),
       new Presenter(std::forward<Args>(args)...));
    it->get()->onClose([this, it]{ _presenters.erase(it); });
    return **it;
  }
};
\end{lstlisting}
\end{frame}

\begin{frame}[fragile,t]{Coleção - Alocação dinâmica - Uso}
\begin{lstlisting}[
    gobble=0,
    linebackgroundcolor={%
      \btLstHL<1>{2,6}%
  }]
struct MainPresenter {
  dyn_presenters<ToolXPresenter> toolxs;
};

void MainPresenter::openToolX()
{ toolxs.instance(view).show(); }
\end{lstlisting}
\end{frame}

\begin{frame}[fragile,t]{Coleção - Por valor - Uso}
\begin{lstlisting}[
    gobble=0,
    linebackgroundcolor={%
      \btLstHL<1>{2}%
  }]
struct MainPresenter {
  presenters<ToolXPresenter> toolxs;
};

void MainPresenter::openToolX()
{ toolxs.instance(view).show(); }
\end{lstlisting}
\end{frame}

\section{Conclusão}

\begin{frame}{Guideline}
  \begin{itemize}
  \item<1-> Classe não cria instâncias próprias
  \item<2-> Classe não é responsável por sua destruição
  \item<3-> Classe não assume se será instanciada dinamicamente ou não
  \item<4-> Procure energicamente por um pai(dono) para o objeto
  \end{itemize}
\end{frame}

\begin{frame}{Obrigado}
  \begin{itemize}
  \item Apresentação
    \begin{itemize}
      \item \href{https://github.com/ricardocosme/presentations/tecgraf/singleton\_suicide}{github.com/ricardocosme/presentations}
    \end{itemize}
  \item Solução genérica
    \begin{itemize}
      \item \href{https://github.com/ricardocosme/saci}{github.com/ricardocosme/saci}
    \end{itemize}
  \end{itemize}
\end{frame}

\end{document}
