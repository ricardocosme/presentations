%!TEX program = xelatex

% Name           : hsrm-beamer-demo.sty
% Author         : Benjamin Weiss (benjamin.weiss@kreatiefton.de)
% Version        : 0.4
% Created on     : 05.05.2013
% Last Edited on : 24.03.2014
% Copyright      : Copyright (c) 2013-2014 by Benjamin Weiss. All rights reserved.
% License        : This file may be distributed and/or modified under the
%                  GNU Public License.
% Description    : HSRM beamer theme demonstration. Also includes a short 
%                  Tutorial regarding the beamer class.

\documentclass[compress]{beamer}
%--------------------------------------------------------------------------
% Common packages
%--------------------------------------------------------------------------
\usepackage[portuguese]{babel}
\usepackage{pgfplots}
\usepackage{array}
\usepackage{graphicx}
\usepackage{multicol}
% Erweiterte Tabellenfunktionen
\usepackage{tabularx,ragged2e}
\usepackage{booktabs}
% Listingserweiterung
\usepackage{pgf, pgffor}
\usepackage{listings}
\usepackage{dirtree}

\usepackage{lstlinebgrd}

%% hack from https://tex.stackexchange.com/questions/451532/recent-issues-with-lstlinebgrd-package-with-listings-after-the-latters-updates
% \makeatletter
% \let\old@lstKV@SwitchCases\lstKV@SwitchCases
% \def\lstKV@SwitchCases#1#2#3{}
% \makeatother
% \usepackage{lstlinebgrd}
% \makeatletter
% \let\lstKV@SwitchCases\old@lstKV@SwitchCases

% \lst@Key{numbers}{none}{%
%     \def\lst@PlaceNumber{\lst@linebgrd}%
%     \lstKV@SwitchCases{#1}%
%     {none:\\%
%      left:\def\lst@PlaceNumber{\llap{\normalfont
%                 \lst@numberstyle{\thelstnumber}\kern\lst@numbersep}\lst@linebgrd}\\%
%      right:\def\lst@PlaceNumber{\rlap{\normalfont
%                 \kern\linewidth \kern\lst@numbersep
%                 \lst@numberstyle{\thelstnumber}}\lst@linebgrd}%
%     }{\PackageError{Listings}{Numbers #1 unknown}\@ehc}}
% \makeatother

\usepackage{hyperref}
\usepackage{fontspec}
\usepackage{siunitx}
\sisetup{math-micro=\text{µ},text-micro=µ}

\makeatletter
%%%%%%%%%%%%%%%%%%%%%%%%%%%%%%%%%%%%%%%%%%%%%%%%%%%%%%%%%%%%%%%%%%%%%%%%%%%%%%
%
% \btIfInRange{number}{range list}{TRUE}{FALSE}
%
% Test in int number <number> is element of a (comma separated) list of ranges
% (such as: {1,3-5,7,10-12,14}) and processes <TRUE> or <FALSE> respectively

\newcount\bt@rangea
\newcount\bt@rangeb

\newcommand\btIfInRange[2]{%
    \global\let\bt@inrange\@secondoftwo%
    \edef\bt@rangelist{#2}%
    \foreach \range in \bt@rangelist {%
        \afterassignment\bt@getrangeb%
        \bt@rangea=0\range\relax%
        \pgfmathtruncatemacro\result{ ( #1 >= \bt@rangea) && (#1 <= \bt@rangeb) }%
        \ifnum\result=1\relax%
            \breakforeach%
            \global\let\bt@inrange\@firstoftwo%
        \fi%
    }%
    \bt@inrange%
}
\newcommand\bt@getrangeb{%
    \@ifnextchar\relax%
        {\bt@rangeb=\bt@rangea}%
        {\@getrangeb}%
}
\def\@getrangeb-#1\relax{%
    \ifx\relax#1\relax%
        \bt@rangeb=100000%   \maxdimen is too large for pgfmath
    \else%
        \bt@rangeb=#1\relax%
    \fi%
}

%%%%%%%%%%%%%%%%%%%%%%%%%%%%%%%%%%%%%%%%%%%%%%%%%%%%%%%%%%%%%%%%%%%%%%%%%%%%%%
%
% \btLstHL<overlay spec>{range list}
%
% TODO BUG: \btLstHL commands can not yet be accumulated if more than one overlay spec match.
% 
\newcommand<>{\btLstHL}[1]{%
  \only#2{\btIfInRange{\value{lstnumber}}{#1}{\color{yellow!20}\def\lst@linebgrdcmd{\color@block}}{\def\lst@linebgrdcmd####1####2####3{}}}%
}%
\makeatother

\lstdefinestyle{emph}{
  % keywordstyle=\color{orange}\bfseries
  linebackgroundcolor=\color{orange!15}
}

\lstset{
literate={~} {$\sim$}{1}, % set tilde as a literal (no process)
escapeinside=@@,
language=c++,
% basicstyle=\small,
basicstyle=\normalsize\sffamily,
keywordstyle=\color{hsrmSec1Dark}\bfseries,
stringstyle=\color{red},
commentstyle=\color{gray},
morecomment=[l][\color{magenta}]{\#},
numbers=left,
numberstyle=\tiny\ttfamily,
stepnumber=1,
showspaces=false,
showstringspaces=false,
showtabs=false,
breaklines=true,
% frame=tb,
framerule=0.5pt,
tabsize=2,
framexleftmargin=0.5em,
framexrightmargin=0.5em,
xleftmargin=0.5em,
xrightmargin=0.5em,
}

%--------------------------------------------------------------------------
% Load theme
%--------------------------------------------------------------------------
\usetheme{hsrm}

% \usepackage{dtklogos} % must be loaded after theme
\usepackage{tikz}
\usetikzlibrary{mindmap,backgrounds}

%--------------------------------------------------------------------------
% General presentation settings
%--------------------------------------------------------------------------
\title{saci::qt::spinbox}
\subtitle{Usando \texttt{coruja::object<T>} como modelos em widgets Qt}
\date{\today}
\author{Ricardo Cosme}
\institute{Instituto Tecgraf de Desenvolvimento de \\ Software Técnico-Científico da PUC-Rio\\{\Medium Tecgraf/PUC-Rio}}

%--------------------------------------------------------------------------
% Notes settings
%--------------------------------------------------------------------------
\setbeameroption{show notes}

\begin{document}
%--------------------------------------------------------------------------
% Titlepage
%--------------------------------------------------------------------------

\maketitle

%\begin{frame}[plain]
%	\titlepage
%\end{frame}

%--------------------------------------------------------------------------
% Table of contents
%--------------------------------------------------------------------------
% \section*{Seções}
% \begin{frame}{Seções}
% 	% hideallsubsections ist empfehlenswert für längere Präsentationen
% 	\tableofcontents[hideallsubsections]
% \end{frame}

%--------------------------------------------------------------------------
% Content
%--------------------------------------------------------------------------
\section{Widgets Qt}

\begin{frame}[fragile,t]{Solução Adhoc \#1 - Modelo não observável}
\lstset{basicstyle=\scriptsize\sffamily}
\begin{lstlisting}[
    gobble=0,
    linebackgroundcolor={%
      \btLstHL<1>{1}%
      \btLstHL<2>{3}%
      \btLstHL<3>{11-13}%
      \btLstHL<4>{15-16,7-8}%
  }]
struct person_t { double height; };

class view_t : public QObject {
  Q_OBJECT
public:
  view_t() : height(&window) {
    QObject::connect(&height, SIGNAL(valueChanged(double)),
                     this, SLOT(height_change_slot(double)));
  }
  virtual ~view_t() = default;
  QMainWindow window;
  QDoubleSpinBox height;
  coruja::signal<void(double)> on_height_change;
public Q_SLOTS:
  void height_change_slot(double v)
  { on_height_change(v); }
};

\end{lstlisting}
\end{frame}

\begin{frame}[fragile,t]{Solução Adhoc \#1 - Modelo não observável}
\lstset{basicstyle=\scriptsize\sffamily}
\begin{lstlisting}[
    gobble=0,
    linebackgroundcolor={%
      \btLstHL<1>{2}%
      \btLstHL<2->{5-6}%
  }]
//inicializa view
view.height.setValue(person.height);

//visão atualiza modelo
_conn = view.on_height_change.connect(
  [&](double v){ person.height = v; });
\end{lstlisting}
\only<3->{
  Problema: múltiplas visões para o modelo
}
\end{frame}

\begin{frame}[fragile,t]{Solução Adhoc \#2 - Modelo observável}
\lstset{basicstyle=\scriptsize\sffamily}
\begin{lstlisting}[
    gobble=0,
    linebackgroundcolor={%
      \btLstHL<1>{1}%
      \btLstHL<2->{8-9}%
  }]
struct person_t { coruja::object<double> height; };

//visão atualiza modelo
_conn = view.on_height_change.connect(
  [&](double v){ person.height = v; });

//inicializa view e modelo atualiza visão
conn_ = person.height.for_each(
  [&](double v){ view.height.setValue(v); });
\end{lstlisting}
\only<3->{
  Problema: Ciclo view -> model -> view
}
\end{frame}

\begin{frame}[fragile,t]{Solução Adhoc \#2 - Modelo observável}
\lstset{basicstyle=\scriptsize\sffamily}
\begin{lstlisting}[
    gobble=0,
    linebackgroundcolor={%
      \btLstHL<1>{1}%
      \btLstHL<2>{5-6}%
  }]
view.height_conn = conn_.get();

void view_t::height_change_slot(double v) {
  if(height_conn != coruja::any_connection{}) {
    auto bc = make_scoped_blocked_connection(height_conn);
    on_height_change(v);
  } else on_height_change(v);
}
\end{lstlisting}
\end{frame}

\begin{frame}[fragile,t]{Solução Adhoc \#2 - Modelo observável}
\lstset{basicstyle=\tiny\sffamily}
\begin{lstlisting}[
    gobble=0,
    linebackgroundcolor={%
      \btLstHL<1>{1}%
      \btLstHL<2>{2-21}%
      \btLstHL<3>{22-26}%
  }]
struct person_t { coruja::object<double> height; };
class view_t : public QObject {
  Q_OBJECT
public:
  view_t() : height(&window) {
    QObject::connect(&height, SIGNAL(valueChanged(double)),
                     this, SLOT(height_change_slot(double)));
  }
  virtual ~view_t() = default;
  QMainWindow window;
  QDoubleSpinBox height;
  coruja::signal<void(double)> on_height_change;
  coruja::any_connection height_conn;
public Q_SLOTS:
  void height_change_slot(double v) {
    if(height_conn != coruja::any_connection{}) {
      auto bc = coruja::make_scoped_blocked_connection(height_conn);
      on_height_change(v);
    } else on_height_change(v);
  }
};
_conn_0 =  person.height.for_each(
  [&](double v){ view.height.setValue(v); });
view.height_conn = conn_0.get();
_conn_1 = view.on_height_change.connect(
  [&](double v){ person.height = v; });
\end{lstlisting}
\end{frame}

\section{saci::qt::spinbox}

\begin{frame}[fragile,t]{Solução Saci}
\lstset{basicstyle=\scriptsize\sffamily}
\begin{lstlisting}[
    gobble=0,
    linebackgroundcolor={%
      \btLstHL<1>{1}%
      \btLstHL<2>{10,12}%
  }]
struct person_t { coruja::object<double> height; };
  
struct view_t {
  QMainWindow window;
private:    
  QDoubleSpinBox _height;
public:    
  view_t(coruja::object<double>& mheight)
    : _height(&window)
    , height(mheight, _height)
  {}
  saci::qt::spinbox height;
};
\end{lstlisting}
\end{frame}

\begin{frame}[fragile,t]{Solução Saci}
\lstset{basicstyle=\scriptsize\sffamily}
\begin{lstlisting}
struct person_t { coruja::object<double> height; };

struct view_t : window_t {
    view_t(coruja::object<double>& mheight)
    : height(mheight, child<QDoubleSpinBox>("height")) {}
    
    saci::qt::spinbox height;
};
\end{lstlisting}
\end{frame}

\begin{frame}[fragile,t]{Solução Saci}
\lstset{basicstyle=\scriptsize\sffamily}
\begin{lstlisting}[
    gobble=0,
    linebackgroundcolor={%
      \btLstHL<1>{1-2}%
      \btLstHL<2->{4-8}%
  }]
template<typename Model>
spinbox::spinbox(Model& model, QDoubleSpinBox& widget)

const QDoubleSpinBox& spinbox::widget() const noexcept
{ return *_widget; }

QDoubleSpinBox& spinbox::widget() noexcept
{ return *_widget; }
\end{lstlisting}
\only<3->{
\texttt{spinbox}, \texttt{checkbox} e \texttt{radio\_btn}, 
}
\end{frame}

\begin{frame}[fragile,t]{coruja::for\_each}
Merge de ObservableObject para uma única reação
\lstset{basicstyle=\scriptsize\sffamily}
\begin{lstlisting}
auto conn = for_each(draw_person_t{}, 
                     person.height, 
                     person.weight);
\end{lstlisting}
\end{frame}

\begin{frame}[fragile,t]{TODO}
\lstset{basicstyle=\scriptsize\sffamily}
\begin{lstlisting}[
    gobble=0,
    linebackgroundcolor={%
      \btLstHL<1->{1-2}%
  }]
template<typename ObservableObjectView>
void spinbox::enable(ObservableObjectView&&)
\end{lstlisting}
\only<2->{
  \begin{itemize}
  \item<1->{Abstração da GUI}
    \begin{itemize}
    \item<2->{Reuso de código ao trocar de engine}
    \end{itemize}
  \end{itemize}
}
\end{frame}

\end{document}
